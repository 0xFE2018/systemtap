% SystemTap Language Reference
\documentclass[twoside,english]{article}
\usepackage{geometry}
\geometry{verbose,letterpaper,tmargin=1.5in,bmargin=1.5in,lmargin=1in,rmargin=1in}
\usepackage{fancyhdr}
\pagestyle{fancy}
\usepackage{array}
\usepackage{varioref}
\usepackage{float}
\usepackage{makeidx}
\usepackage{verbatim}
\usepackage{url}
\makeindex

\makeatletter

%%%%%%%%%%%%%%%%%%%%%%%%%%%%%% LyX specific LaTeX commands.
\newcommand{\noun}[1]{\textsc{#1}}
%% Bold symbol macro for standard LaTeX users
%\providecommand{\boldsymbol}[1]{\mbox{\boldmath $#1$}}

%% Because html converters don't know tabularnewline
\providecommand{\tabularnewline}{\\}

%%%%%%%%%%%%%%%%%%%%%%%%%%%%%% User specified LaTeX commands.
\setlength{\parindent}{0pt}
%\setlength{\parskip}{3pt plus 2pt minus 1pt}
\setlength{\parskip}{5pt}

%
% this makes list spacing much better.
%
\newenvironment{my_itemize}{
\begin{itemize}
  \setlength{\itemsep}{1pt}
  \setlength{\parskip}{0pt}
  \setlength{\parsep}{0pt}}{\end{itemize}
}

\newenvironment{vindent}
{\begin{list}{}{\setlength{\listparindent}{6pt}}
\item[]}
{\end{list}}

\usepackage{babel}
\makeatother
\begin{document}

\title{SystemTap Language Reference}

\maketitle
\newpage{}
This document was derived from other documents contributed to the SystemTap project by employees of Red Hat, IBM and Intel.\newline

Copyright \copyright\space  2007 Red Hat Inc.\newline
Copyright \copyright\space  2007-2009 IBM Corp.\newline
Copyright \copyright\space  2007 Intel Corporation.\newline

Permission is granted to copy, distribute and/or modify this document
under the terms of the GNU Free Documentation License, Version 1.2
or any later version published by the Free Software Foundation;
with no Invariant Sections, no Front-Cover Texts, and no Back-Cover Texts.\newline

The GNU Free Documentation License is available from 
\url{http://www.gnu.org/licenses/fdl.html} or by writing to
the Free Software Foundation, Inc., 51 Franklin Street,
Fifth Floor, Boston, MA 02110-1301, USA.
\newpage{}
\tableofcontents{}
\listoftables
\newpage{}

\section{SystemTap overview\label{sec:SystemTap-Overview}}

\subsection{About this guide}

This guide is a comprehensive reference of SystemTap's language constructs
and syntax. The contents borrow heavily from existing SystemTap documentation
found in manual pages and the tutorial. The presentation of information here
provides the reader with a single place to find language syntax and recommended
usage. In order to successfully use this guide, you should be familiar with
the general theory and operation of SystemTap. If you are new to SystemTap,
you will find the tutorial to be an excellent place to start learning. For
detailed information about tapsets, see the manual pages provided with the
distribution. For information about the entire collection of SystemTap reference
material, see Section~\ref{sec:For-Further-Reference} 

\subsection{Reasons to use SystemTap}

SystemTap provides infrastructure to simplify the gathering of information
about a running Linux kernel so that it may be further analyzed. This analysis
assists in identifying the underlying cause of a performance or functional
problem. SystemTap was designed to eliminate the need for a developer to
go through the tedious instrument, recompile, install, and reboot sequence
normally required to collect this kind of data. To do this, it provides a
simple command-line interface and scripting language for writing kernel instrumentation.
With SystemTap, developers, system administrators, and users can easily write
scripts that gather and manipulate kernel data that is not otherwise available
using standard Linux tools. Users of SystemTap will find it to be a significant
improvement over older methods.

\subsection{Event-action language}
\index{language}
SystemTap's language is strictly typed, declaration free, procedural, and
inspired by dtrace and awk. Source code points or events in the kernel are
associated with handlers, which are subroutines that are executed synchronously.
These probes are conceptually similar to \char`\"{}breakpoint command lists\char`\"{}
in the GDB debugger.

There are two main outermost constructs: probes and functions. Within these,
statements and expressions use C-like operator syntax and precedence.

\subsection{Sample SystemTap scripts}
\index{example scripts}
Following are some example scripts that illustrate the basic operation of
SystemTap. For more examples, see the examples/small\_demos/ directory in
the source directory, the SystemTap wiki at \url{http://sourceware.org/systemtap/wiki/HomePage},
or the SystemTap War Stories at \url{http://sourceware.org/systemtap/wiki/WarStories} page.

\subsubsection{Basic SystemTap syntax and control structures}

The following code examples demonstrate SystemTap syntax and control structures.

\begin{vindent}
\begin{verbatim}
global odds, evens

probe begin {
    # "no" and "ne" are local integers
    for (i = 0; i < 10; i++) {
        if (i % 2) odds [no++] = i
            else evens [ne++] = i
    }

    delete odds[2]
    delete evens[3]
    exit()
}

probe end {
    foreach (x+ in odds)
        printf ("odds[%d] = %d", x, odds[x])

    foreach (x in evens-)
        printf ("evens[%d] = %d", x, evens[x])
}
\end{verbatim}
\end{vindent}
This prints:

\begin{vindent}
\begin{verbatim}
odds[0] = 1
odds[1] = 3
odds[3] = 7
odds[4] = 9
evens[4] = 8
evens[2] = 4
evens[1] = 2
evens[0] = 0
\end{verbatim}
\end{vindent}
Note that all variable types are inferred, and that all locals and globals
are initialized.

\subsubsection{Primes between 0 and 49}

\begin{vindent}
\begin{verbatim}
function isprime (x) {
    if (x < 2) return 0
    for (i = 2; i < x; i++) {
        if (x % i == 0) return 0
        if (i * i > x) break
    }
    return 1
}

probe begin {
    for (i = 0; i < 50; i++)
        if (isprime (i)) printf("%d\n", i)
    exit()
}
\end{verbatim}
\end{vindent}
This prints:

\begin{vindent}
\begin{verbatim}
2
3
5
7
11
13
17
19
23
29
31
37
41
43
47
\end{verbatim}
\end{vindent}

\subsubsection{Recursive functions}
\index{recursion}
\begin{vindent}
\begin{verbatim}
function fibonacci(i) {
    if (i < 1) error ("bad number")
    if (i == 1) return 1
    if (i == 2) return 2
    return fibonacci (i-1) + fibonacci (i-2)
}

probe begin {
    printf ("11th fibonacci number: %d", fibonacci (11))
    exit ()
}
\end{verbatim}
\end{vindent}
This prints:

\begin{vindent}
\begin{verbatim}
11th fibonacci number: 118
\end{verbatim}
\end{vindent}
Any larger number input to the function may exceed the MAXACTION or MAXNESTING
limits, which will be caught at run time and result in an error. For more
about limits see Section~\ref{sub:SystemTap-safety}.
\newpage{}
\subsection{The stap command}
\index{stap}
The stap program is the front-end to the SystemTap tool. It accepts probing
instructions written in its scripting language, translates those instructions
into C code, compiles this C code, and loads the resulting kernel module
into a running Linux kernel to perform the requested system trace or probe
functions. You can supply the script in a named file, from standard input,
or from the command line. The program runs until it is interrupted by the
user or a sufficient number of soft errors, or if the script voluntarily
invokes the exit() function.

The stap command does the following:

\begin{itemize}
\item Translates the script
\item Generates and compiles a kernel module
\item Inserts the module; output to stap's stdout
\item CTRL-C unloads the module and terminates stap
\end{itemize}
For a full list of options to the stap command, see the stap(1) manual page.

\subsection{Safety and security\label{sub:SystemTap-safety}}
\index{limits}
SystemTap is an administrative tool. It exposes kernel internal data structures
and potentially private user information. It requires root privileges to
actually run the kernel objects it builds using the \textbf{sudo} command,
applied to the \textbf{staprun} program.

staprun is a part of the SystemTap package, dedicated to module loading and
unloading and kernel-to-user data transfer. Since staprun does not perform
any additional security checks on the kernel objects it is given, do not
give elevated privileges via sudo to untrusted users.

The translator asserts certain safety constraints. \index{constraints}It
ensures that no handler routine can run for too long, allocate memory, perform
unsafe operations, or unintentionally interfere with the kernel. Use of script
global variables is locked to protect against manipulation by concurrent
probe handlers. Use of \emph{guru mode} constructs such as embedded C (see
Section~\ref{sub:Embedded-C}) can violate these constraints, leading to
a kernel crash or data corruption.

The resource use limits are set by macros in the generated C code. These
may be overridden with the -D flag. The following list describes a selection
of these macros:

\textbf{MAXNESTING} -- The maximum number of recursive function call levels. The default is 10.

\textbf{MAXSTRINGLEN} -- The maximum length of strings. The default is 128.

\textbf{MAXTRYLOCK} -- The maximum number of iterations to wait for locks on global variables before
declaring possible deadlock and skipping the probe. The default is 1000.

\textbf{MAXACTION} -- The maximum number of statements to execute during any single probe hit. The default is 1000. 

\textbf{MAXMAPENTRIES} -- The maximum number of rows in an array if the array size is not specified
explicitly when declared. The default is 2048.

\textbf{MAXERRORS} -- The maximum number of soft errors before an exit is triggered. The default is 0.

\textbf{MAXSKIPPED} -- The maximum number of skipped reentrant probes before an exit is triggered. The default is 100.

\textbf{MINSTACKSPACE} -- The minimum number of free kernel stack bytes required in order to run a
probe handler. This number should be large enough for the probe handler's
own needs, plus a safety margin.  The default is 1024.

If something goes wrong with stap or staprun after a probe has started running,
you may safely kill both user processes, and remove the active probe kernel
module with the rmmod command. Any pending trace messages may be lost.

\section{Types of SystemTap scripts\label{sec:Types-of-SystemTap}}

\subsection{Probe scripts}

Probe scripts are analogous to programs; these scripts identify probe points
and associated handlers.

\subsection{Tapset scripts}

Tapset scripts are libraries of probe aliases and auxiliary functions.

The /usr/share/systemtap/tapset directory contains tapset scripts. While
these scripts look like regular SystemTap scripts, they cannot be run directly.

\section{Components of a SystemTap script}

The main construct in the scripting language identifies probes. Probes associate
abstract events with a statement block, or probe handler, that is to be executed
when any of those events occur.

The following example shows how to trace entry and exit from a function using
two probes.

\begin{vindent}
\begin{verbatim}
probe kernel.function("sys_mkdir") { log ("enter") }
probe kernel.function("sys_mkdir").return { log ("exit") }
\end{verbatim}
\end{vindent}

To list the probe-able functions in the kernel, use the last-pass option
to the translator. The output needs to be filtered because each inlined function
instance is listed separately. The following statement is an example.

\begin{vindent}
\begin{verbatim}
# stap -p2 -e 'probe kernel.function("*") {}' | sort | uniq
\end{verbatim}
\end{vindent}

\subsection{Probe definitions}

The general syntax is as follows.

\begin{vindent}
\begin{verbatim}
probe PROBEPOINT [, PROBEPOINT] { [STMT ...] }
\end{verbatim}
\end{vindent}
Events are specified in a special syntax called \emph{probe points}. There
are several varieties of probe points defined by the translator, and tapset
scripts may define others using aliases. The provided probe points are listed
in the stapprobes(5) man pages.

The probe handler is interpreted relative to the context of each event. For
events associated with kernel code, this context may include variables defined
in the source code at that location. These \emph{target variables}\index{target variables}
are presented to the script as variables whose names are prefixed with a
dollar sign (\$). They may be accessed only if the compiler used to compile
the kernel preserved them, despite optimization. This is the same constraint
imposed by a debugger when working with optimized code. Other events may
have very little context.


\subsection{Probe aliases\label{sub:Probe-aliases}}
\index{probe aliases}
The general syntax is as follows.

\begin{vindent}
\begin{verbatim}
probe <alias> = <probepoint> { <prologue_stmts> }
probe <alias> += <probepoint> { <epilogue_stmts> }
\end{verbatim}
\end{vindent}
New probe points may be defined using \emph{aliases}. A probe point alias
looks similar to probe definitions, but instead of activating a probe at
the given point, it defines a new probe point name as an alias to an existing
one. New probe aliases may refer to one or more existing probe aliases. 
Multiple aliases may share the same name. The following is an example.

\begin{vindent}
\begin{verbatim}
probe socket.sendmsg = kernel.function ("sock_sendmsg") { ... }
probe socket.do_write = kernel.function ("do_sock_write") { ... }
probe socket.send = socket.sendmsg, socket.do_write { ... }
\end{verbatim}
\end{vindent}
There are two types of aliases, the prologue style and the epilogue style
which are identified by the equal sign (\texttt{\textbf{=}}) and \char`\"{}\texttt{\textbf{+=}}\char`\"{}
respectively.

A probe that names the new probe point will create an actual probe, with
the handler of the alias \emph{pre-pended}.

This pre-pending behavior serves several purposes. It allows the alias definition
to pre-process the context of the probe before passing control to the handler
specified by the user. This has several possible uses, demonstrated as follows.

\begin{vindent}
\begin{verbatim}
# Skip probe unless given condition is met:
if ($flag1 != $flag2) next

# Supply values describing probes:
name = "foo"

# Extract the target variable to a plain local variable:
var = $var
\end{verbatim}
\end{vindent}

\subsubsection{Prologue-style aliases (=)}
\index{prologue-style aliases}
\index{=}
For a prologue style alias, the statement block that follows an alias definition
is implicitly added as a prologue to any probe that refers to the alias.
The following is an example.

\begin{vindent}
\begin{verbatim}
# Defines a new probe point syscall.read, which expands to
# kernel.function("sys_read"), with the given statement as
# a prologue.
#
probe syscall.read = kernel.function("sys_read") {
    fildes = $fd
}
\end{verbatim}
\end{vindent}

\subsubsection{Epilogue-style aliases (+=)}
\index{epilogue-style aliases}
\index{+=}
The statement block that follows an alias definition is implicitly added
as an epilogue to any probe that refers to the alias.  It is not useful
to define new variable there (since no subsequent code will see it), but
rather the code can take action based upon variables left set by the
prologue or by the user code.  The following is an example:

\begin{vindent}
\begin{verbatim}
# Defines a new probe point with the given statement as an
# epilogue.
#
probe syscall.read += kernel.function("sys_read") {
    if (traceme) println ("tracing me")
}
\end{verbatim}
\end{vindent}

\subsubsection{Probe alias usage}

A probe alias is used the same way as any built-in probe type, by
naming it:

\begin{vindent}
\begin{verbatim}
probe syscall.read {
    printf("reading fd=%d\n", fildes)
}
\end{verbatim}
\end{vindent}

\subsubsection{Unused alias variables}
\index{unused variables}
An unused alias variable is a variable defined in a probe alias, usually
as one of a group of \texttt{var = \$var} assignments, which is not actually
used by the script probe that instantiates the alias. These variables are
discarded.

\subsection{Variables\label{sub:Variables}}
\index{variables}
Identifiers for variables and functions are alphanumeric sequences, and may
include the underscore (\_) and the dollar sign (\$) characters. They may
not start with a plain digit. Each variable is by default local to the probe
or function statement block where it is mentioned, and therefore its scope
and lifetime is limited to a particular probe or function invocation. Scalar
variables are implicitly typed as either string or integer. Associative arrays
also have a string or integer value, and a tuple of strings or integers serves
as a key. Arrays must be declared as global. Local arrays\index{local arrays}
are not allowed.

The translator performs \emph{type inference} on all identifiers, including
array indexes and function parameters. Inconsistent type-related use of identifiers
results in an error.

Variables may be declared global. Global variables are shared among all probes
and remain instantiated as long as the SystemTap session. There is one namespace
for all global variables, regardless of the script file in which they are
found. Because of possible concurrency limits, such as multiple probe handlers,
each global variable used by a probe is automatically read- or write-locked
while the handler is running. A global declaration may be written at the
outermost level anywhere in a script file, not just within a block of code.
Global variables which are written but never read will be displayed
automatically at session shutdown.  The following declaration marks 
\texttt{var1} and \texttt{var2} as global.
The translator will infer a value type for each, and if the variable is used
as an array, its key types.

\begin{vindent}
\begin{verbatim}
global var1[=<value>], var2[=<value>]
\end{verbatim}
\end{vindent}

\subsection{Auxiliary functions\label{sub:Auxiliary-functions}}
\index{auxiliary functions}
General syntax:

\begin{vindent}
\begin{verbatim}
function <name>[:<type>] ( <arg1>[:<type>], ... ) { <stmts> }
\end{verbatim}
\end{vindent}
SystemTap scripts may define subroutines to factor out common work. Functions
may take any number of scalar arguments, and must return a single scalar
value. Scalars in this context are integers or strings. For more information
on scalars, see Section~\ref{sub:Variables} and Section~\ref{sub:Data-types}\texttt{.}
The following is an example function declaration.

\begin{vindent}
\begin{verbatim}
function thisfn (arg1, arg2) {
    return arg1 + arg2
}
\end{verbatim}
\end{vindent}
Note the general absence of type declarations, which are inferred by the
translator. If desired, a function definition may include explicit type declarations
for its return value, its arguments, or both. This is helpful for embedded-C
functions. In the following example, the type inference engine need only
infer the type of arg2, a string.

\begin{vindent}
\begin{verbatim}
function thatfn:string(arg1:long, arg2) {
    return sprintf("%d%s", arg1, arg2)
}
\end{verbatim}
\end{vindent}
Functions may call others or themselves recursively, up to a fixed nesting
limit. See Section~\ref{sub:SystemTap-safety}.


\subsection{Embedded C\label{sub:Embedded-C}}
\index{embedded C}
SystemTap supports a \emph{guru\index{guru mode} mode} where script safety
features such as code and data memory reference protection are removed. Guru
mode is set by passing the ''-g'' flag to the stap command. When in guru
mode, the translator accepts embedded code enclosed between {}``\%\{''
and {}``\%\}'' markers in the script file. Embedded code is transcribed
verbatim, without analysis, in sequence, into generated C code. At the outermost
level of a script, guru mode may be useful to add \#include instructions,
or any auxiliary definitions for use by other embedded code.


\subsection{Embedded C functions}

General syntax:

\begin{vindent}
\begin{verbatim}
function <name>:<type> ( <arg1>:<type>, ... ) %{ <C_stmts> %}
\end{verbatim}
\end{vindent}
Embedded code is permitted in a function body. In that case, the script language
body is replaced entirely by a piece of C code enclosed between \%\{ and
\%\} markers. The enclosed code may do anything reasonable and safe as allowed
by the parser.

There are a number of undocumented but complex safety constraints on concurrency,
resource consumption and runtime limits that are applied to code written
in the SystemTap language. These constraints are not applied to embedded
C code, so use such code with caution as it is used verbatim. Be especially
careful when dereferencing pointers. Use the kread() macro to dereference
any pointers that could potentially be invalid or dangerous. If you are unsure,
err on the side of caution and use kread(). The kread() macro is one of the
safety mechanisms used in code generated by embedded C. It protects against
pointer accesses that could crash the system.

For example, to access the pointer chain \texttt{name = skb->dev->name} in
embedded C, use the following code.

\begin{vindent}
\begin{verbatim}
struct net_device *dev;
char *name;
dev = kread(&(skb->dev));
name = kread(&(dev->name));
\end{verbatim}
\end{vindent}
The memory locations reserved for input and output values are provided to
a function using a macro named \texttt{THIS}\index{THIS}. The following
are examples.

\begin{vindent}
\begin{verbatim}
function add_one (val) %{
    THIS->__retvalue = THIS->val + 1;
}
function add_one_str (val) %{
    strlcpy (THIS->__retvalue, THIS->val, MAXSTRINGLEN);
    strlcat (THIS->__retvalue, "one", MAXSTRINGLEN);
}
\end{verbatim}
\end{vindent}
The function argument and return value types must be inferred by the translator
from the call sites in order for this method to work. You should examine
C code generated for ordinary script language functions to write compatible
embedded-C. Note that all SystemTap functions and probes run with interrupts
disabled, thus you cannot call functions that might sleep from within embedded
C.

\section{Probe points\label{sec:Probe-Points}}
\index{probe points}
\subsection{General syntax}
\index{probe syntax}
The general probe point syntax is a dotted-symbol sequence. This divides
the event namespace into parts, analogous to the style of the Domain Name
System. Each component identifier is parameterized by a string or number
literal, with a syntax analogous to a function call.

The following are all syntactically valid probe points.

\begin{vindent}
\begin{verbatim}
kernel.function("foo")
kernel.function("foo").return
module{"ext3"}.function("ext3_*")
kernel.function("no_such_function") ?
syscall.*
end
timer.ms(5000)
\end{verbatim}
\end{vindent}
Probes may be broadly classified into \emph{synchronous}\index{synchronous}
or \emph{asynchronous}.\index{asynchronous} A synchronous event occurs when
any processor executes an instruction matched by the specification. This
gives these probes a reference point (instruction address) from which more
contextual data may be available. Other families of probe points refer to
asynchronous events such as timers, where no fixed reference point is related.
Each probe point specification may match multiple locations, such as by using
wildcards or aliases, and all are probed. A probe declaration may contain
several specifications separated by commas, which are all probed.

\subsubsection{Prefixes}
\index{prefixes}
Prefixes specify the probe target, such as \textbf{kernel}, \textbf{module},
\textbf{timer}, and so on.

\subsubsection{Suffixes}
\index{suffixes}
Suffixes further qualify the point to probe, such as \textbf{.return} for the
exit point of a probed function. The absence of a suffix implies the function 
entry point.

\subsubsection{Wildcarded file names, function names}
\index{wildcards}
A component may include an asterisk ({*}) character, which expands to other
matching probe points. An example follows.

\begin{vindent}
\begin{verbatim}
kernel.syscall.*
kernel.function("sys_*)
\end{verbatim}
\end{vindent}

\subsubsection{Optional probe points\label{sub:Optional-probe-points}}
\index{?}
A probe point may be followed by a question mark (?) character, to indicate
that it is optional, and that no error should result if it fails to expand.
This effect passes down through all levels of alias or wildcard expansion.

The following is the general syntax.

\begin{vindent}
\begin{verbatim}
kernel.function("no_such_function") ?
\end{verbatim}
\end{vindent}

\subsection{Built-in probe point types (DWARF probes)}
\index{built-in probes}
\index{dwarf probes}
This family of probe points uses symbolic debugging information for the target
kernel or module, as may be found in executables that have not
been stripped, or in the separate \textbf{debuginfo} packages. They allow
logical placement of probes into the execution path of the target 
by specifying a set of points in the source or object code. When a matching
statement executes on any processor, the probe handler is run in that context.

Points in a kernel are identified by module, source file, line number, function
name or some combination of these.

Here is a list of probe point specifications currently supported: 

\begin{vindent}
\begin{verbatim}
kernel.function(PATTERN)
kernel.function(PATTERN).call
kernel.function(PATTERN).return
kernel.function(PATTERN).return.maxactive(VALUE)
kernel.function(PATTERN).inline
kernel.function(PATTERN).label(LPATTERN)
module(MPATTERN).function(PATTERN)
module(MPATTERN).function(PATTERN).call
module(MPATTERN).function(PATTERN).return.maxactive(VALUE)
module(MPATTERN).function(PATTERN).inline
kernel.statement(PATTERN)
kernel.statement(ADDRESS).absolute
module(MPATTERN).statement(PATTERN)
\end{verbatim}
\end{vindent}

The \textbf{.function} variant places a probe near the beginning of the named
function, so that parameters are available as context variables. 

The \textbf{.return} variant places a probe at the moment of return from the named
function, so the return value is available as the \$return context variable.
The entry parameters are also available, though the function may have changed
their values.  Return probes may be further qualified with \textbf{.maxactive}, 
which specifies how many instances of the specified function can be probed simultaneously.
You can leave off \textbf{.maxactive} in most cases, as the default should be sufficient.
However, if you notice an excessive number of skipped probes, try setting \textbf{.maxactive}
to incrementally higher values to see if the number of skipped probes decreases.

The \textbf{.inline} modifier for \textbf{.function} filters the results to include only 
instances of inlined functions. The \textbf{.call} modifier selects the opposite subset.
Inline functions do not have an identifiable return point, so \textbf{.return}
is not supported on \textbf{.inline} probes.

The \textbf{.statement} variant places a probe at the exact spot, exposing those local
variables that are visible there.

In the above probe descriptions, MPATTERN stands for a string literal
that identifies the loaded kernel module of interest and LPATTERN
stands for a source program label. Both MPATTERN and LPATTERN may
include asterisk ({*}), square brackets \char`\"{}{[}]\char`\"{}, and
question mark (?) wildcards.

PATTERN stands for a string literal that identifies a point in the program.
It is composed of three parts:

\begin{enumerate}
\item The first part is the name of a function, as would appear in the nm program's
output. This part may use the asterisk and question mark wildcard operators
to match multiple names.
\item The second part is optional, and begins with the ampersand (@) character.
It is followed by the path to the source file containing the function,
which may include a wildcard pattern, such as mm/slab{*}.
In most cases, the path should be relative to the top of the
linux source directory, although an absolute path may be necessary for some kernels.
If a relative pathname doesn't work, try absolute.
\item The third part is optional if the file name part was given. It identifies
the line number in the source file, preceded by a ``:'' or ``+''.   
The line number is assumed to be an
absolute line number if preceded by a ``:'', or relative to the entry of
the function if preceded by a ``+''.
All the lines in the function can be matched with ``:*''.
A range of lines x through y can be matched with ``:x-y''.

\end{enumerate}
Alternately, specify PATTERN as a numeric constant to indicate a relative
module address or an absolute kernel address.

Some of the source-level variables, such as function parameters, locals,
or globals visible in the compilation unit, are visible to probe handlers.
Refer to these variables by prefixing their name with a dollar sign within
the scripts. In addition, a special syntax allows limited traversal of structures,
pointers, and arrays.

\texttt{\$var} refers to an in-scope variable var. If it is a type similar
to an integer, it will be cast to a 64-bit integer for script use. Pointers
similar to a string (char {*}) are copied to SystemTap string values by the
\texttt{kernel\_string()} or \texttt{user\_string()} functions.

\texttt{\$var->field} traverses a structure's field. The indirection operator
may be repeated to follow additional levels of pointers.

\texttt{\$var{[}N]} indexes into an array. The index is given with a literal
number.

\texttt{\$\$vars} expands to a character string that is equivalent to
\texttt{sprintf("parm1=\%x ... parmN=\%x var1=\%x ... varN=\%x", \$parm1, ..., \$parmN,
\$var1, ..., \$varN)}

\texttt{\$\$locals} expands to a character string that is equivalent to
\texttt{sprintf("var1=\%x ... varN=\%x", \$var1, ..., \$varN)}

\texttt{\$\$parms} expands to a character string that is equivalent to
\texttt{sprintf("parm1=\%x ... parmN=\%x", \$parm1, ..., \$parmN)}


\subsubsection{kernel.function, module().function}
\index{kernel.function}
\index{module().function}
The \textbf{.function} variant places a probe near the beginning of the named function,
so that parameters are available as context variables.

General syntax:

\begin{vindent}
\begin{verbatim}
kernel.function("func[@file]"
module("modname").function("func[@file]"
\end{verbatim}
\end{vindent}
Examples:

\begin{vindent}
\begin{verbatim}
# Refers to all kernel functions with "init" or "exit"
# in the name:
kernel.function("*init*"), kernel.function("*exit*")

# Refers to any functions within the "kernel/sched.c"
# file that span line 240:
kernel.function("*@kernel/sched.c:240")

# Refers to all functions in the ext3 module:
module("ext3").function("*")
\end{verbatim}
\end{vindent}

\subsubsection{kernel.statement, module().statement}
\index{kernel.statement}
\index{module().statement}
The \textbf{.statement} variant places a probe at the exact spot, exposing those local
variables that are visible there.

General syntax:

\begin{vindent}
\begin{verbatim}
kernel.statement("func@file:linenumber")
module("modname").statement("func@file:linenumber")
\end{verbatim}
\end{vindent}
Example:

\begin{vindent}
\begin{verbatim}
# Refers to the statement at line 2917 within the
# kernel/sched.c file:
kernel.statement("*@kernel/sched.c:2917")
# Refers to the statement at line bio_init+3 within the fs/bio.c file:
kernel.statement("bio_init@fs/bio.c+3")
\end{verbatim}
\end{vindent}


\subsection{DWARF-less probing}
\index{DWARF-less probing}

In the absence of debugging information, you can still use the
\emph{kprobe} family of probes to examine the entry and exit points of
kernel and module functions. You cannot look up the arguments or local
variables of a function using these probes. However, you can access
the parameters by following this procedure:

When you're stopped at the entry to a function, you can refer to the 
function's arguments by number. For example, when probing the function 
declared:

\begin{vindent}
\begin{verbatim}
asmlinkage ssize_t sys_read(unsigned int fd, char __user * buf, size_t
count)
\end{verbatim}
\end{vindent}

You can obtain the values of \texttt{fd}, \texttt{buf}, and
\texttt{count}, respectively, as \texttt{uint\_arg(1)},
\texttt{pointer\_arg(2)}, and \texttt{ulong\_arg(3)}. In this case, your
probe code must first call \texttt{asmlinkage()}, because on some
architectures the asmlinkage attribute affects how the function's
arguments are passed.

When you're in a return probe, \texttt{\$return} isn't supported
without DWARF, but you can call \texttt{returnval()} to get the value
of the register in which the function value is typically returned, or
call \texttt{returnstr()} to get a string version of that value.

And at any code probepoint, you can call
\texttt{{register("regname")}} to get the value of the specified CPU
register when the probe point was hit.
\texttt{u\_register("regname")} is like \texttt{register("regname")},
but interprets the value as an unsigned integer.

SystemTap supports the following constructs:
\begin{vindent}
\begin{verbatim}
kprobe.function(FUNCTION)
kprobe.function(FUNCTION).return
kprobe.module(NAME).function(FUNCTION)
kprobe.module(NAME).function(FUNCTION).return
kprobe.statement(ADDRESS).absolute
\end{verbatim}
\end{vindent}

Use \textbf{.function} probes for kernel functions and
\textbf{.module} probes for probing functions of a specified module.
If you do not know the absolute address of a kernel or module
function, use \textbf{.statement} probes. Do not use wildcards in
\textit{FUNCTION} and \textit{MODULE} names. Wildcards cause the probe
to not register. Also, run statement probes in guru mode only.


\begin{comment}
\subsection{Marker probes}

This family of probe points connects to static probe markers inserted into
the kernel or a module. These markers are special macro calls in the kernel
that make probing faster and more reliable than with DWARF-based probes.
DWARF debugging information is not required to use probe markers.

Marker probe points begin with a kernel or module(\char`\"{}\emph{name}\char`\"{})
prefix, the same as DWARF probes. This prefix identifies the source of the
symbol table used for finding markers. The suffix names the marker itself:
mark(\char`\"{}\emph{name}\char`\"{}). The marker name string, which may
contain wildcard characters, is matched against the names given to the marker
macros when the kernel or module was compiled.

The handler associated with a marker probe reads any optional parameters
specified at the macro call site named \$arg1 through \$argNN, where NN is
the number of parameters supplied by the macro. Number and string parameters
are passed in a type-safe manner.
\end{comment}

\subsection{Timer probes}
\index{timer probes}
You can use intervals defined by the standard kernel jiffies\index{jiffies}
timer to trigger probe handlers asynchronously. A \emph{jiffy} is a kernel-defined
unit of time typically between 1 and 60 msec. Two probe point variants are
supported by the translator: 

\begin{vindent}
\begin{verbatim}
timer.jiffies(N)
timer.jiffies(N).randomize(M)
\end{verbatim}
\end{vindent}
The probe handler runs every N jiffies. If the \texttt{randomize}\index{randomize}
component is given, a linearly distributed random value in the range {[}-M
\ldots{} +M] is added to N every time the handler executes. N is restricted
to a reasonable range (1 to approximately 1,000,000), and M is restricted
to be less than N. There are no target variables provided in either context.
Probes can be run concurrently on multiple processors.

Intervals may be specified in units of time. There are two probe point variants
similar to the jiffies timer:

\begin{vindent}
\begin{verbatim}
timer.ms(N)
timer.ms(N).randomize(M)
\end{verbatim}
\end{vindent}
Here, N and M are specified in milliseconds\index{milliseconds}, but the
full options for units are seconds (s or sec), milliseconds (ms or msec),
microseconds (us or usec), nanoseconds (ns or nsec), and hertz (hz). Randomization
is not supported for hertz timers.

The resolution of the timers depends on the target kernel. For kernels prior
to 2.6.17, timers are limited to jiffies resolution, so intervals are rounded
up to the nearest jiffies interval. After 2.6.17, the implementation uses
hrtimers for tighter precision, though the resulting resolution will be dependent
upon architecture. In either case, if the randomize component is given, then
the random value will be added to the interval before any rounding occurs.

Profiling timers are available to provide probes that execute on all CPUs
at each system tick. This probe takes no parameters, as follows.

\begin{vindent}
\begin{verbatim}
timer.profile
\end{verbatim}
\end{vindent}
Full context information of the interrupted process is available, making
this probe suitable for implementing a time-based sampling profiler.

The following is an example of timer usage.

\begin{vindent}
\begin{verbatim}
# Refers to a periodic interrupt, every 1000 jiffies:
timer.jiffies(1000)

# Fires every 5 seconds:
timer.sec(5)

# Refers to a periodic interrupt, every 1000 +/- 200 jiffies:
timer.jiffies(1000).randomize(200)
\end{verbatim}
\end{vindent}

\subsection{Return probes}
\index{return probes}
The \texttt{.return} variant places a probe at the moment of return from
the named function, so that the return value is available as the \$return
context variable. The entry parameters are also accessible in the context
of the return probe, though their values may have been changed by the function.
Inline functions do not have an identifiable return point, so \texttt{.return}
is not supported on \texttt{.inline} probes.


\subsection{Special probe points}

The probe points \texttt{begin} and \texttt{end} are defined by the translator
to refer to the time of session startup and shutdown. There are no target
variables available in either context.


\subsubsection{begin}
\index{begin}
The \texttt{begin} probe is the start of the SystemTap session. All \texttt{begin}
probe handlers are run during the startup of the session. All global variables
must be declared prior to this point.


\subsubsection{end}
\index{end}
The \texttt{end} probe is the end of the SystemTap session. All \texttt{end}
probes are run during the normal shutdown of a session, such as in the aftermath
of an \texttt{exit} function call, or an interruption from the user. In the
case of an shutdown triggered by error, \texttt{end} probes are not run.


\subsubsection{begin and end probe sequence}
\index{sequence}
\texttt{begin} and \texttt{end} probes are specified with an optional sequence
number that controls the order in which they are run. If no sequence number
is provided, the sequence number defaults to zero and probes are run in the
order that they occur in the script file. Sequence numbers may be either
positive or negative, and are especially useful for tapset writers who want
to do initialization in a \texttt{begin} probe. The following are examples.

\begin{vindent}
\begin{verbatim}
# In a tapset file:
probe begin(-1000) { ... }

# In a user script:
probe begin { ... }
\end{verbatim}
\end{vindent}
The user script \texttt{begin} probe defaults to sequence number zero, so
the tapset \texttt{begin} probe will run first.


\subsubsection{never}
\index{never}
The \texttt{never} probe point is defined by the translator to mean \emph{never}.
Its statements are analyzed for symbol and type correctness, but its probe
handler is never run. This probe point may be useful in conjunction with
optional probes. See Section~\ref{sub:Optional-probe-points}.


\begin{comment} % Comment out until perfmon code is reactivated
\subsection{Probes to monitor performance}

The perfmon family of probe points is used to access the performance monitoring
hardware available in modern processors. These probe points require perfmon2
support in the kernel to access the hardware.

Performance monitor hardware points have a \texttt{perfmon} prefix. The suffix
names the event being counted, for example \texttt{counter(event)}. The event
names are specific to the processor implementation, except for generic cycle
and instructions events, which are available on all processors. The probe
\texttt{perfmon.counter(event)} starts a counter on the processor which counts
the number of events that occur on that processor. For more details about
the performance monitoring events available on a specific processor, see
the help text returned by typing the perfmon2 command \texttt{pfmon -l.}

\subsubsection{\$counter}

\$counter is a handle used in the body of a probe for operations involving
the counter associated with the probe.

\subsubsection{read\_counter}

read\_counter is a function passed to the handle for a perfmon probe. It
returns the current count for the event.
\end{comment}

\section{Language elements\label{sec:Language-Elements}}


\subsection{Identifiers}
\index{identifiers}
\emph{Identifiers} are used to name variables and functions. They are an
alphanumeric sequence that may include the underscore (\_) and dollar sign
(\$) characters. They have the same syntax as C identifiers, except that
the dollar sign is also a legal character. Identifiers that begin with a
dollar sign are interpreted as references to variables in the target software,
rather than to SystemTap script variables. Identifiers may not start with
a plain digit. 


\subsection{Data types\label{sub:Data-types}}
\index{data types}
The SystemTap language includes a small number of data types, but no type
declarations. A variable's type is inferred\index{inference} from its use.
To support this inference, the translator enforces consistent typing of function
arguments and return values, array indices and values. There are no implicit
type conversions between strings and numbers. Inconsistent type-related use
of identifiers signals an error.


\subsubsection{Literals}
\index{literals}
Literals are either strings or integers. Literals can be expressed as decimal,
octal, or hexadecimal, using C notation. Type suffixes (e.g., \emph{L} or
\emph{U}) are not used. 


\subsubsection{Integers\label{sub:Integers}}
\index{integers} \index{numbers}
Integers are decimal, hexadecimal, or octal, and use the same notation as
in C. Integers are 64-bit signed quantities, although the parser also accepts
(and wraps around) values above positive $2^{63}$ but below $2^{64}$.


\subsubsection{Strings\label{sub:Strings}}
\index{strings}
Strings are enclosed in quotation marks ({}``string''), and pass through
standard C escape codes with backslashes. Strings are limited in length to
MAXSTRINGLEN. For more information about this and other limits, see Section~\ref{sub:SystemTap-safety}.


\subsubsection{Associative arrays}

See Section~\ref{sec:Associative-Arrays}


\subsubsection{Statistics}

See Section~\ref{sec:Statistics}


\subsection{Semicolons}
\index{;}
The semicolon is the null statement, or do nothing statement. It is optional,
and useful as a separator between statements to improve detection of syntax
errors and to reduce ambiguities in grammar.


\subsection{Comments}
\index{comments}
Three forms of comments are supported, as follows.

\begin{vindent}
\begin{verbatim}
# ... shell style, to the end of line
// ... C++ style, to the end of line
/* ... C style ... */
\end{verbatim}
\end{vindent}

\subsection{Whitespace}
\index{whitespace}
As in C, spaces, tabs, returns, newlines, and comments are treated as whitespace.
Whitespace is ignored by the parser.


\subsection{Expressions}
\index{expressions}
SystemTap supports a number of operators that use the same general syntax,
semantics, and precedence as in C and awk. Arithmetic is performed per C
rules for signed integers. If the parser detects division by zero or an overflow,
it generates an error. The following subsections list these operators.


\subsubsection{Binary numeric operators}
\index{binary}
\texttt{{*} / \% + - >\,{}> <\,{}< \& \textasciicircum{}
| \&\& ||}


\subsubsection{Binary string operators}
\index{binary}
\texttt{\textbf{.}} (string concatenation)


\subsubsection{Numeric assignment operators}
\index{numeric}
\texttt{= {*}= /= \%= += -= >\,{}>= <\,{}<=
\&= \textasciicircum{}= |=}


\subsubsection{String assignment operators}

\texttt{= .=}


\subsubsection{Unary numeric operators}
\index{unary}
\texttt{+ - ! \textasciitilde{} ++ -{}-}


\subsubsection{Binary numeric or string comparison operators}
\index{comparison}
\texttt{< > <= >= == !=}


\subsubsection{Ternary operator\label{sub:Ternary-operator}}
\index{?}
\texttt{cond ? exp1 : exp2}


\subsubsection{Grouping operator}
\index{grouping}
\texttt{( exp )}


\subsubsection{Function call}
\index{fn}
General syntax:

\texttt{fn ({[} arg1, arg2, ... ])}


\subsubsection{\$ptr-\textgreater member}
\index{pointer}
\texttt{ptr} is a kernel pointer available in a probed context.


\subsubsection{\textless value\textgreater\ in \textless array\_name\textgreater}
\index{index}
This expression evaluates to true if the array contains an element with the
specified index.


\subsubsection{{[} \textless value\textgreater, ... ] in \textless array\_name\textgreater}

The number of index values must match the number of indexes previously specified.


\subsection{Literals passed in from the stap command line\label{sub:Literals-passed-in}}
\index{literals}
\emph{Literals} are either strings enclosed in double quotes ('' '') or
integers. For information about integers, see Section~\ref{sub:Integers}.
For information about strings, see Section~\ref{sub:Strings}.

Script arguments at the end of a command line are expanded as literals. You
can use these in all contexts where literals are accepted. A reference to
a nonexistent argument number is an error.


\subsubsection{\$1 \ldots{} \$\textless NN\textgreater\ for integers}
\index{\$}
Use \texttt{\$1 \ldots{} \$<NN>} for casting as a numeric literal.


\subsubsection{@1 \ldots{} @\textless NN\textgreater\ for strings}

Use \texttt{@1 \ldots{} @<NN>} for casting as a string literal.


\subsubsection{Examples}

For example, if the following script named example.stp

\begin{vindent}
\begin{verbatim}
probe begin { printf("%d, %s\n", $1, @2) }
\end{verbatim}
\end{vindent}
is invoked as follows

\begin{vindent}
\begin{verbatim}
# stap example.stp 10 mystring
\end{verbatim}
\end{vindent}
then 10 is substituted for \$1 and \char`\"{}mystring\char`\"{} for @2. The
output will be

\begin{vindent}
\begin{verbatim}
10, mystring
\end{verbatim}
\end{vindent}

\subsection{Conditional compilation}


\subsubsection{Conditions}
\index{conditions}
One of the steps of parsing is a simple conditional preprocessing stage.
The general form of this is similar to the ternary operator (Section~\ref{sub:Ternary-operator}).

\begin{vindent}
\begin{verbatim}
%( CONDITION %? TRUE-TOKENS %)
%( CONDITION %? TRUE-TOKENS %: FALSE-TOKENS %)
\end{verbatim}
\end{vindent}
The CONDITION is a limited expression whose format is determined by its first
keyword. The following is the general syntax.

\begin{vindent}
\begin{verbatim}
%( <condition> %? <code> [ %: <code> ] %)
\end{verbatim}
\end{vindent}

\subsubsection{Conditions based on kernel version: kernel\_v, kernel\_vr}
\index{kernel version}
\index{kernel\_vr}
\index{kernel\_v}
If the first part of a conditional expression is the identifier \texttt{kernel\_v}
or \texttt{kernel\_vr}, the second part must be one of six standard numeric
comparison operators {}``\textless'', {}``\textless ='', {}``=='', {}``!='', {}``\textgreater'',
or {}``\textgreater ='',
and the third part must be a string literal that contains an RPM-style version-release
value. The condition returns true if the version of the target kernel (as
optionally overridden by the \textbf{-r} option) matches the given version
string. The comparison is performed by the glibc function strverscmp.

\texttt{kernel\_v} refers to the kernel version number only, such as {}``2.6.13\char`\"{}.

\texttt{kernel\_vr} refers to the kernel version number including the release
code suffix, such as {}``2.6.13-1.322FC3smp''.


\subsubsection{Conditions based on architecture: arch}
\index{arch}
If the first part of the conditional expression is the identifier \texttt{arch}
which refers to the processor architecture, then the second part is a string
comparison operator ''=='' or ''!='', and the third part is a string
literal for matching it. This comparison is a simple string equality or inequality.
The currently supported architecture strings are i386, i686, x86\_64, ia64,
s390x and ppc64.


\subsubsection{True and False Tokens}
\index{tokens}
TRUE-TOKENS and FALSE-TOKENS are zero or more general parser tokens, possibly
including nested preprocessor conditionals, that are pasted into the input
stream if the condition is true or false. For example, the following code
induces a parse error unless the target kernel version is newer than 2.6.5.

\begin{vindent}
\begin{verbatim}
%( kernel_v <= "2.6.5" %? **ERROR** %) # invalid token sequence
\end{verbatim}
\end{vindent}
The following code adapts to hypothetical kernel version drift.

\begin{vindent}
\begin{verbatim}
probe kernel.function (
    %( kernel_v <= "2.6.12" %? "__mm_do_fault" %:
        %( kernel_vr == "2.6.13-1.8273FC3smp" %? "do_page_fault" %: UNSUPPORTED %)
    %)) { /* ... */ }

%( arch == "ia64" %?
    probe syscall.vliw = kernel.function("vliw_widget") {}
%)
\end{verbatim}
\end{vindent}

\section{Statement types\label{sec:Statement-Types}}

Statements enable procedural control flow within functions and probe handlers.
The total number of statements executed in response to any single probe event
is limited to MAXACTION, which defaults to 1000. See Section~\ref{sub:SystemTap-safety}.


\subsection{break and continue}
\index{break}
\index{continue}
Use \texttt{break} or \texttt{continue} to exit or iterate the innermost
nesting loop statement, such as within a \texttt{while, for,} or \texttt{foreach}
statement. The syntax and semantics are the same as those used in C.


\subsection{delete}
\index{delete}
\texttt{delete} removes an element.

The following statement removes from ARRAY the element specified by the index
tuple. The value will no longer be available, and subsequent iterations will
not report the element. It is not an error to delete an element that does
not exist.

\begin{vindent}
\begin{verbatim}
delete ARRAY[INDEX1, INDEX2, ...]
\end{verbatim}
\end{vindent}
The following syntax removes all elements from ARRAY:

\begin{vindent}
\begin{verbatim}
delete ARRAY
\end{verbatim}
\end{vindent}
The following statement removes the value of SCALAR. Integers and strings
are cleared to zero and null (\char`\"{}\char`\"{}) respectively, while statistics
are reset to their initial empty state.

\begin{vindent}
\begin{verbatim}
delete SCALAR
\end{verbatim}
\end{vindent}

\subsection{do}
\index{do}
The \texttt{do} statement has the same syntax and semantics as in C.

\begin{vindent}
\begin{verbatim}
do STMT while (EXP)
\end{verbatim}
\end{vindent}

\subsection{EXP (expression)}
\index{expression}
An \texttt{expression} executes a string- or integer-valued expression and
discards the value.


\subsection{for}
\index{for}
General syntax:
\begin{vindent}
\begin{verbatim}
for (EXP1; EXP2; EXP3) STMT
\end{verbatim}
\end{vindent}
The \texttt{for} statement is similar to the \texttt{for} statement in C.
The \texttt{for} expression executes EXP1 as initialization. While EXP2 is
non-zero, it executes STMT, then the iteration expression EXP3.

\subsection{foreach\label{sub:foreach}}
\index{foreach}
General syntax:
\begin{vindent}
\begin{verbatim}
foreach (VAR in ARRAY) STMT
\end{verbatim}
\end{vindent}
The \texttt{foreach} statement loops over each element of a named global array, assigning
the current key to VAR. The array must not be modified within the statement.
If you add a single plus (+) or minus (-) operator after the VAR or the ARRAY
identifier, the iteration order will be sorted by the ascending or descending
index or value. 

The following statement behaves the same as the first example, except it
is used when an array is indexed with a tuple of keys.  Use a sorting suffix
on at most one VAR or ARRAY identifier.

\begin{vindent}
\begin{verbatim}
foreach ([VAR1, VAR2, ...] in ARRAY) STMT
\end{verbatim}
\end{vindent}
The following statement is the same as the first example, except that the
\texttt{limit} keyword limits the number of loop iterations to EXP times.
EXP is evaluated once at the beginning of the loop.

\begin{vindent}
\begin{verbatim}
foreach (VAR in ARRAY limit EXP) STMT
\end{verbatim}
\end{vindent}

\subsection{if}
\index{if}
General syntax:

\begin{vindent}
\begin{verbatim}
if (EXP) STMT1 [ else STMT2 ]
\end{verbatim}
\end{vindent}
The \texttt{if} statement compares an integer-valued EXP to zero. It executes
the first STMT if non-zero, or the second STMT if zero.

The \texttt{if} command has the same syntax and semantics as used in C.


\subsection{next}
\index{next}
The \texttt{next} statement returns immediately from the enclosing probe
handler.


\subsection{; (null statement)}
\index{;}
\index{null statement}
General syntax:

\begin{vindent}
\begin{verbatim}
statement1
;
statement2
\end{verbatim}
\end{vindent}
The semicolon represents the null statement, or do nothing. It is useful
as an optional separator between statements to improve syntax error detection
and to handle certain grammar ambiguities.


\subsection{return}
\index{return}
General syntax:

\begin{vindent}
\begin{verbatim}
return EXP
\end{verbatim}
\end{vindent}
The \texttt{return} statement returns the EXP value from the enclosing function.
If the value of the function is not returned, then a return statement is
not needed, and the function will have a special \emph{unknown} type with
no return value.

\subsection{\{ \} (statement block)}
\index{\{ \}}
\index{statement block}
This is the statement block with zero or more statements enclosed within
brackets. The following is the general syntax:

\begin{vindent}
\begin{verbatim}
{ STMT1 STMT2 ... }
\end{verbatim}
\end{vindent}
The statement block executes each statement in sequence in the block. Separators
or terminators are generally not necessary between statements. The statement
block uses the same syntax and semantics as in C.


\subsection{while}
\index{while}
General syntax:

\begin{vindent}
\begin{verbatim}
while (EXP) STMT
\end{verbatim}
\end{vindent}
The \texttt{while} statement uses the same syntax and semantics as in C.
In the statement above, while the integer-valued EXP evaluates to non-zero,
the parser will execute STMT. 


\section{Associative arrays\label{sec:Associative-Arrays}}
\index{associative arrays}
Associative arrays are implemented as hash tables with a maximum size set
at startup. Associative arrays are too large to be created dynamically for
individual probe handler runs, so they must be declared as global. The basic
operations for arrays are setting and looking up elements. These operations
are expressed in awk syntax: the array name followed by an opening bracket
({[}), a comma-separated list of up to five index index expressions, and
a closing bracket (]). Each index expression may be a string or a number,
as long as it is consistently typed throughout the script.


\subsection{Examples}

\begin{vindent}
\begin{verbatim}
# Increment the named array slot:
foo [4,"hello"] ++

# Update a statistic:
processusage [uid(),execname()] ++

# Set a timestamp reference point:
times [tid()] = get_cycles()

# Compute a timestamp delta:
delta = get_cycles() - times [tid()]
\end{verbatim}
\end{vindent}

\subsection{Types of values}

Array elements may be set to a number or a string. The type must be consistent
throughout the use of the array. The first assignment to the array defines
the type of the elements. Unset array elements may be fetched and return
a null value (zero or empty string) as appropriate, but they are not seen
by a membership test.


\subsection{Array capacity}

Array sizes can be specified explicitly or allowed to default to the maximum
size as defined by MAXMAPENTRIES. See Section~\ref{sub:SystemTap-safety}
for details on changing MAXMAPENTRIES.

You can explicitly specify the size of an array as follows:

\begin{vindent}
\begin{verbatim}
global ARRAY[<size>]
\end{verbatim}
\end{vindent}
If you do not specify the size parameter, then the array is created to hold
MAXMAPENTRIES number of elements


\subsection{Iteration, foreach}
\index{foreach}
Like awk, SystemTap's foreach creates a loop that iterates over key tuples
of an array, not only values. The iteration may be sorted by any single key
or a value by adding an extra plus symbol (+) or minus symbol (-) to the
code. The following are examples.

\begin{vindent}
\begin{verbatim}
# Simple loop in arbitrary sequence:
foreach ([a,b] in foo)
    fuss_with(foo[a,b])

# Loop in increasing sequence of value:
foreach ([a,b] in foo+) { ... }

# Loop in decreasing sequence of first key:
foreach ([a-,b] in foo) { ... }
\end{verbatim}
\end{vindent}
The \texttt{break} and \texttt{continue} statements also work inside foreach
loops. Since arrays can be large but probe handlers must execute quickly,
you should write scripts that exit iteration early, if possible. For simplicity,
SystemTap forbids any modification of an array during iteration with a foreach.


\section{Statistics (aggregates)\label{sec:Statistics}}
\index{aggregates}
Aggregate instances are used to collect statistics on numerical values, when
it is important to accumulate new data quickly and in large volume. These
instances operate without exclusive locks, and store only aggregated stream
statistics. Aggregates make sense only for global variables. They are stored
individually or as elements of an array.

\subsection{The aggregation (\textless\hspace{1 sp}\textless\hspace{1 sp}\textless) operator}
\index{\textless\hspace{1 sp}\textless\hspace{1 sp}\textless}
The aggregation operator is {}``\textless\hspace{1 sp}\textless\hspace{1 sp}\textless'',
and its effect is similar to an assignment or a C++ output streaming operation.
The left operand specifies a scalar or array-index \emph{l-value}, which
must be declared global. The right operand is a numeric expression. The meaning
is intuitive: add the given number to the set of numbers to compute their
statistics. The specific list of statistics to gather is given separately
by the extraction functions. The following is an example.

\begin{vindent}
\begin{verbatim}
a <<< delta_timestamp
writes[execname()] <<< count
\end{verbatim}
\end{vindent}

\subsection{Extraction functions}
\index{extraction}
For each instance of a distinct extraction function operating on a given
identifier, the translator computes a set of statistics. With each execution
of an extraction function, the aggregation is computed for that moment across
all processors. The first argument of each function is the same style of
l-value as used on the left side of the aggregation operation.


\subsection{Integer extractors}

The following functions provide methods to extract information about integer
values.


\subsubsection{@count(s)}
\index{count}
This statement returns the number of all values accumulated into s.


\subsubsection{@sum(s)}
\index{sum}
This statement returns the total of all values accumulated into s.


\subsubsection{@min(s)}
\index{min}
This statement returns the minimum of all values accumulated into s.


\subsubsection{@max(s)}
\index{max}
This statement returns the maximum of all values accumulated into s.


\subsubsection{@avg(s)}
\index{avg}
This statement returns the average of all values accumulated into s.


\subsection{Histogram extractors}
\index{histograms}
The following functions provide methods to extract histogram information.
Printing a histogram with the print family of functions renders a histogram
object as a tabular "ASCII art" bar chart.

\subsubsection{@hist\_linear}
\index{hist\_linear}
The statement \texttt{@hist\_linear(v,L,H,W)} represents a linear histogram
\texttt{v}, where \emph{L} and \emph{H} represent the lower and upper end of
a range of values and \emph{W} represents the width (or size) of each bucket
within the range.  The low and high values can be negative, but the overall
difference (high minus low) must be positive. The width parameter must also
be positive.

In the output, a range of consecutive empty buckets may be replaced with a tilde
(\textasciitilde{}) character.  This can be controlled on the command line
with -DHIST\_ELISION=\textless\hspace{1 sp}num\textgreater\hspace{1 sp},
where \textless\hspace{1 sp}num\textgreater\hspace{1 sp} specifies how many
empty buckets at the top and bottom of the range to print.
The default is 2.  A \textless\hspace{1 sp}num\textgreater\hspace{1 sp} of 0
removes all empty buckets. A negative \textless\hspace{1 sp}num\textgreater\hspace{1 sp}
turns off bucket removal all together.

For example, if you specify -DHIST\_ELISION=3 and the histogram has 10 
consecutive empty buckets, the first 3 and last 3 empty buckets will
be printed and the middle 4 empty buckets will be represented by a
tilde (\textasciitilde{}).

The following is an example.

\begin{vindent}
\begin{verbatim}
global reads
probe netdev.receive {
    reads <<< length
}
probe end {
    print(@hist_linear(reads, 0, 10240, 200))
}
\end{verbatim}
\end{vindent}
This generates the following output.

\pagebreak
\begin{vindent}
\begin{verbatim}
value |-------------------------------------------------- count
    0 |@@@@@@@@@@@@@@@@@@@@@@@@@@@@@@@@@@@@@@@@@@@@@@@@@@ 1650
  200 |                                                      8
  400 |                                                      0
  600 |                                                      0
      ~
 1000 |                                                      0
 1200 |                                                      0
 1400 |                                                      1
 1600 |                                                      0
 1800 |                                                      0
\end{verbatim}
\end{vindent}
This shows that 1650 network reads were of a size between 0 and 200 bytes, 
8 reads were between 200 and 400 bytes, and 1 read was between
1200 and 1400 bytes.  The tilde (\textasciitilde{}) character indicates 
buckets 700, 800 and 900 were removed because they were empty.
Empty buckets at the upper end were also removed.

\subsubsection{@hist\_log}
\index{hist\_log}
The statement \texttt{@hist\_log(v)} represents a base-2 logarithmic 
histogram.  Empty buckets are replaced with a tilde (\textasciitilde{})
character in the same way as \texttt{@hist\_linear()} (see above).

The following is an example.

\begin{vindent}
\begin{verbatim}
global reads
probe netdev.receive {
    reads <<< length
}
probe end {
    print(@hist_log(reads))
}
\end{verbatim}
\end{vindent}
This generates the following output.

\begin{vindent}
\begin{verbatim}
value |-------------------------------------------------- count
    8 |                                                      0
   16 |                                                      0
   32 |                                                    254
   64 |                                                      3
  128 |                                                      2
  256 |                                                      2
  512 |                                                      4
 1024 |@@@@@@@@@@@@@@@@@@@@@@@@@@@@@@@@@@@@@@@@@@@@@@@@@ 16689
 2048 |                                                      0
 4096 |                                                      0
\end{verbatim}
\end{vindent}

\section{Predefined functions\label{sec:Predefined-Functions}}

Unlike built-in functions, predefined functions are implemented in tapsets.


\subsection{Output functions}

The following sections describe the functions you can use to output data.


\subsubsection{error}
\index{error}
General syntax:

\begin{vindent}
\begin{verbatim}
error:unknown (msg:string)
\end{verbatim}
\end{vindent}
This function logs the given string to the error stream. It appends an implicit
end-of-line. It blocks any further execution of statements in this probe.
If the number of errors exceeds the MAXERRORS parameter, it triggers an \texttt{exit}.


\subsubsection{log}
\index{log}
General syntax:

\begin{vindent}
\begin{verbatim}
log:unknown (msg:string)
log (const char *fmt, )
\end{verbatim}
\end{vindent}
This function logs data. \texttt{log} sends the message immediately to staprun
and to the bulk transport (relayfs) if it is being used. If the last character
given is not a newline, then one is added.

This function is not as efficient as printf and should only be used for urgent
messages.

\subsubsection{print}
\index{print}
General syntax:

\begin{vindent}
\begin{verbatim}
print:unknown ()
\end{verbatim}
\end{vindent}
This function prints a single value of any type.


\subsubsection{printf}
\index{printf}
General syntax:

\begin{vindent}
\begin{verbatim}
printf:unknown (fmt:string, )
\end{verbatim}
\end{vindent}
The printf function takes a formatting string as an argument, and a number
of values of corresponding types, and prints them all. The format must be a
literal string constant. The printf formatting directives are similar to those
of C, except that they are fully checked for type by the translator.

The formatting string can contain tags that are defined as follows:

\begin{vindent}
\begin{verbatim}
%[flags][width][.precision][length]specifier
\end{verbatim}
\end{vindent}
Where \texttt{specifier} is required and defines the type and the interpretation
of the value of the corresponding argument. The following table shows the
details of the specifier parameter:

\begin{table}[H]
\caption{printf specifier values}
\begin{tabular}{|>{\raggedright}p{1in}|>{\raggedright}p{3.5in}|>{\raggedright}p{1.25in}|}
\hline
\textbf{Specifier}&
\textbf{Output}&
\textbf{Example}\tabularnewline
\hline
\hline 
d or i&
Signed decimal&
392\tabularnewline
\hline 
o&
Unsigned octal&
610\tabularnewline
\hline 
s&
String&
sample\tabularnewline
\hline 
u&
Unsigned decimal&
7235\tabularnewline
\hline 
x&
Unsigned hexadecimal (lowercase letters)&
7fa\tabularnewline
\hline 
X&
Unsigned hexadecimal (uppercase letters)&
7FA\tabularnewline
\hline 
p&
Pointer address&
0x0000000000bc614e\tabularnewline
\hline 
b&
Writes a binary value as text. The field width specifies the number of bytes
to write. Valid specifications are \%b, \%1b, \%2b, \%4b and \%8b. The default
width is 8 (64-bits).&
See below\tabularnewline
\hline 
\%&
A \% followed by another \% character will write \% to stdout.&
\%\tabularnewline
\hline
\end{tabular}
\end{table}
The tag can also contain \texttt{flags}, \texttt{width}, \texttt{.precision}
and \texttt{modifiers} sub-specifiers, which are optional and follow these
specifications:

\begin{table}[H]
\caption{printf flag values}
\begin{tabular}{|>{\raggedright}p{1.5in}|>{\raggedright}p{4.5in}|}
\hline
\textbf{Flags}&
\textbf{Description}\tabularnewline
\hline
\hline
- (minus sign)&
Left-justify within the given field width. Right justification is the default
(see \texttt{width} sub-specifier).\tabularnewline
\hline 
+ (plus sign)&
Precede the result with a plus or minus sign even for positive numbers. By
default, only negative numbers are preceded with a minus sign.\tabularnewline
\hline 
(space)&
If no sign is going to be written, a blank space is inserted before the value.\tabularnewline
\hline 
\#&
Used with \texttt{o}, \texttt{x} or \texttt{X} specifiers the value is preceded
with \texttt{0}, \texttt{0x} or \texttt{0X} respectively for non-zero values.\tabularnewline
\hline 
0&
Left-pads the number with zeroes instead of spaces, where padding is specified
(see \texttt{width} sub-specifier).\tabularnewline
\hline
\end{tabular}
\end{table}

\begin{table}[H]
\caption{printf width values}
\begin{tabular}{|>{\raggedright}p{1.5in}|>{\raggedright}p{4.5in}|}
\hline
\textbf{Width}&
\textbf{Description}\tabularnewline
\hline
\hline
(number)&
Minimum number of characters to be printed. If the value to be printed is
shorter than this number, the result is padded with blank spaces. The value
is not truncated even if the result is larger.\tabularnewline
\hline
\end{tabular}
\end{table}

%
\begin{table}[H]

\caption{printf precision values}

\begin{tabular}{|>{\raggedright}p{1.5in}|>{\raggedright}p{4.5in}|}
\hline 
\textbf{Precision}&
\textbf{Description}\tabularnewline
\hline
\hline 
.number&
For integer specifiers (\texttt{d, i, o, u, x, X}): \texttt{precision} specifies
the minimum number of digits to be written. If the value to be written is
shorter than this number, the result is padded with leading zeros. The value
is not truncated even if the result is longer. A precision of 0 means that
no character is written for the value 0. For s: this is the maximum number
of characters to be printed. By default all characters are printed until
the ending null character is encountered. When no \texttt{precision} is specified,
the default is 1. If the period is specified without an explicit value for
\texttt{precision}, 0 is assumed.\tabularnewline
\hline
\end{tabular}
\end{table}

\textbf{Binary Write Examples}

The following is an example of using the binary write functions:

\begin{vindent}
\begin{verbatim}
probe begin {
    for (i = 97; i < 110; i++)
        printf("%3d: %1b%1b%1b\n", i, i, i-32, i-64)
    exit()
}
\end{verbatim}
\end{vindent}
This prints:

\begin{vindent}
\begin{verbatim}
 97: aA!
 98: bB"
 99: cC#
100: dD$
101: eE%
102: fF&
103: gG'
104: hH(
105: iI)
106: jJ*
107: kK+
108: lL,
109: mM-
\end{verbatim}
\end{vindent}
Another example:

\begin{vindent}
\begin{verbatim}
stap -e 'probe begin{printf("%b%b", 0xc0dedbad, \
0x12345678);exit()}' | hexdump -C

\end{verbatim}
\end{vindent}
This prints:

\begin{vindent}
\begin{verbatim}
00000000  ad db de c0 00 00 00 00  78 56 34 12 00 00 00 00  |........xV4.....|
00000010
\end{verbatim}
\end{vindent}
Another example:

\begin{vindent}
\begin{verbatim}
probe begin{
    printf("%1b%1b%1blo %1b%1brld\n", 72,101,108,87,111)
    exit()
}
\end{verbatim}
\end{vindent}
This prints:

\begin{vindent}
\begin{verbatim}
Hello World
\end{verbatim}
\end{vindent}

\subsubsection{printd}
\index{printd}
General syntax:

\begin{vindent}
\begin{verbatim}
printd:unknown (delimiter:string, )
\end{verbatim}
\end{vindent}
This function takes a string delimiter and two or more values of any type, then
prints the values with the delimiter interposed. The delimiter must be a
literal string constant.

For example:
\begin{vindent}
\begin{verbatim}
printd("/", "one", "two", "three", 4, 5, 6)
\end{verbatim}
\end{vindent}
prints:
\begin{vindent}
\begin{verbatim}
one/two/three/4/5/6
\end{verbatim}
\end{vindent}

\subsubsection{printdln}
\index{printdln}
General syntax:

\begin{vindent}
\begin{verbatim}
printdln:unknown ()
\end{verbatim}
\end{vindent}
This function operates like \texttt{printd}, but also appends a newline.

\subsubsection{println}
\index{println}
General syntax:

\begin{vindent}
\begin{verbatim}
println:unknown ()
\end{verbatim}
\end{vindent}
This function operates like \texttt{print}, but also appends a newline.

\subsubsection{sprint}
\index{sprint}
General syntax:

\begin{vindent}
\begin{verbatim}
sprint:unknown ()
\end{verbatim}
\end{vindent}
This function operates like \texttt{print}, but returns the string rather
than printing it.

\subsubsection{sprintf}
\index{sprintf}
General syntax:

\begin{vindent}
\begin{verbatim}
sprintf:unknown (fmt:string, )
\end{verbatim}
\end{vindent}
This function operates like \texttt{printf}, but returns the formatted string
rather than printing it.


\subsubsection{system}
\index{system}
General syntax:

\begin{vindent}
\begin{verbatim}
system (cmd:string)
\end{verbatim}
\end{vindent}
The system function runs a command on the system. The specified command runs
in the background once the current probe completes. 


\subsubsection{warn}
\index{warn}
General syntax:

\begin{vindent}
\begin{verbatim}
warn:unknown (msg:string)
\end{verbatim}
\end{vindent}
This function sends a warning message immediately to staprun. It is also
sent over the bulk transport (relayfs) if it is being used. If the last character
is not a newline, then one is added.

\subsection{Context at the probe point}

The following functions provide ways to access the current task context 
at a probe point. Note that these may not return correct values when
a probe is hit in interrupt context.

\subsubsection{backtrace}
\index{backtrace}
General syntax:

\begin{vindent}
\begin{verbatim}
backtrace:string ()
\end{verbatim}
\end{vindent}
Returns a string of hex addresses that are a backtrace of the
stack. The output is truncated to MAXSTRINGLEN.

\subsubsection{caller}
\index{caller}
General syntax:

\begin{vindent}
\begin{verbatim}
caller:string()
\end{verbatim}
\end{vindent}
Returns the address and name of the calling function. It works
only for return probes.

\subsubsection{caller\_addr}
\index{caller\_addr}
General syntax:

\begin{vindent}
\begin{verbatim}
caller_addr:long ()
\end{verbatim}
\end{vindent}
Returns the address of the calling function. It works only
for return probes.


\subsubsection{cpu}
\index{cpu}
General syntax:

\begin{vindent}
\begin{verbatim}
cpu:long ()
\end{verbatim}
\end{vindent}
Returns the current cpu number.


\subsubsection{egid}
\index{egid}
General syntax:

\begin{vindent}
\begin{verbatim}
egid:long ()
\end{verbatim}
\end{vindent}
Returns the effective group ID of the current process.


\subsubsection{euid}
\index{euid}
General syntax:

\begin{vindent}
\begin{verbatim}
euid:long ()
\end{verbatim}
\end{vindent}
Returns the effective user ID of the current process.


\subsubsection{execname}
\index{execname}
General syntax:

\begin{vindent}
\begin{verbatim}
execname:string ()
\end{verbatim}
\end{vindent}
Returns the name of the current process.


\subsubsection{gid}
\index{gid}
General syntax:

\begin{vindent}
\begin{verbatim}
gid:long ()
\end{verbatim}
\end{vindent}
Returns the group ID of the current process.


\subsubsection{is\_return}
\index{is\_return}
General syntax:

\begin{vindent}
\begin{verbatim}
is_return:long ()
\end{verbatim}
\end{vindent}
Returns 1 if the probe point is a return probe, else it returns
zero.

\noun{Deprecated}.


\subsubsection{pexecname}
\index{pexecname}
General syntax:

\begin{vindent}
\begin{verbatim}
pexecname:string ()
\end{verbatim}
\end{vindent}
Returns the name of the parent process.


\subsubsection{pid}
\index{pid}
General syntax:

\begin{vindent}
\begin{verbatim}
pid:long ()
\end{verbatim}
\end{vindent}
Returns the process ID of the current process.


\subsubsection{ppid}
\index{ppid}
General syntax:

\begin{vindent}
\begin{verbatim}
ppid:long ()
\end{verbatim}
\end{vindent}
Returns the process ID of the parent process.


\subsubsection{tid}
\index{tid}
General syntax:

\begin{vindent}
\begin{verbatim}
tid:long ()
\end{verbatim}
\end{vindent}
Returns the ID of the current thread.


\subsubsection{uid}
\index{uid}
General syntax:

\begin{vindent}
\begin{verbatim}
uid:long ()
\end{verbatim}
\end{vindent}
Returns the user ID of the current task.


\subsubsection{print\_backtrace}
\index{print\_backtrace}
General syntax:

\begin{vindent}
\begin{verbatim}
print_backtrace:unknown ()
\end{verbatim}
\end{vindent}
This function is equivalent to \texttt{print\_stack(backtrace())}, except
that deeper stack nesting is supported. The function does not return a value.


\subsubsection{print\_regs}
\index{print\_regs}
General syntax:

\begin{vindent}
\begin{verbatim}
print_regs:unknown ()
\end{verbatim}
\end{vindent}
This function prints a register dump.


\subsubsection{print\_stack}
\index{print\_stack}
General syntax:

\begin{vindent}
\begin{verbatim}
print_stack:unknown (stk:string)
\end{verbatim}
\end{vindent}
This function performs a symbolic lookup of the addresses in the given string,
which is assumed to be the result of a prior call to \texttt{backtrace()}.
It prints one line per address. Each printed line includes the address, the
name of the function containing the address, and an estimate of its position
within that function. The function does not return a value.


\subsubsection{stack\_size}
\index{stack\_size}
General syntax:

\begin{vindent}
\begin{verbatim}
stack_size:long ()
\end{verbatim}
\end{vindent}
Returns the size of the stack.


\subsubsection{stack\_unused}
\index{stack\_unused}
General syntax:

\begin{vindent}
\begin{verbatim}
stack_unused:long ()
\end{verbatim}
\end{vindent}
Returns how many bytes are currently unused in the stack.


\subsubsection{stack\_used}
\index{stack\_used}
General syntax:

\begin{vindent}
\begin{verbatim}
stack_used:long ()
\end{verbatim}
\end{vindent}
Returns how many bytes are currently used in the stack.


\subsubsection{stp\_pid}
\index{stp\_pid}
\begin{vindent}
\begin{verbatim}
stp_pid:long ()
\end{verbatim}
\end{vindent}
Returns the process ID of the of the staprun process.


\subsubsection{target}
\index{target}
General syntax:

\begin{vindent}
\begin{verbatim}
target:long ()
\end{verbatim}
\end{vindent}
Returns the process ID of the target process. This is useful
in conjunction with the -x PID or -c CMD command-line options to stap. An
example of its use is to create scripts that filter on a specific process.

\begin{verbatim}
-x <pid>
\end{verbatim}
target() returns the pid specified by -x

\begin{verbatim}
-c <command>
\end{verbatim}
target() returns the pid for the executed command specified
by -c.

\subsection{Task data}

These functions return data about a task.  They all require a task handle as
input, such as the  value return by task\_current() or the variables
prev\_task and next\_task in the scheduler.ctxswitch probe alias.

\subsubsection{task\_cpu}
\index{task\_cpu}
General syntax:

\begin{vindent}
\begin{verbatim}
task_cpu:long (task:long)
\end{verbatim}
\end{vindent}
Returns the scheduled cpu for the given task.


\subsubsection{task\_current}
\index{task\_current}
General syntax:

\begin{vindent}
\begin{verbatim}
task_current:long ()
\end{verbatim}
\end{vindent}
Returns the address of the task\_struct representing
the current process. This address can be passed to the various task\_{*}()
functions to extract more task-specific data.


\subsubsection{task\_egid}
\index{task\_egid}
General syntax:

\begin{vindent}
\begin{verbatim}
task_egid:long (task:long)
\end{verbatim}
\end{vindent}
Returns the effective group ID of the given task.


\subsubsection{task\_execname}
\index{task\_execname}
General syntax:

\begin{vindent}
\begin{verbatim}
task_execname:string (task:long)
\end{verbatim}
\end{vindent}
Returns the name of the given task.


\subsubsection{task\_euid}
\index{task\_euid}
General syntax:

\begin{vindent}
\begin{verbatim}
task_euid:long (task:long)
\end{verbatim}
\end{vindent}
Returns the effective user ID of the given task.


\subsubsection{task\_gid}
\index{task\_gid}
General syntax:

\begin{vindent}
\begin{verbatim}
task_gid:long (task:long)
\end{verbatim}
\end{vindent}
Returns the group ID of the given task.


\subsubsection{task\_nice}
\index{task\_nice}
General syntax:

\begin{vindent}
\begin{verbatim}
task_nice:long (task:long)
\end{verbatim}
\end{vindent}
Returns the nice value of the given task.


\subsubsection{task\_parent}
\index{task\_parent}
General syntax:

\begin{vindent}
\begin{verbatim}
task_parent:long (task:long)
\end{verbatim}
\end{vindent}
Returns the address of the parent task\_struct of the given
task. This address can be passed to the various task\_{*}() functions to
extract more task-specific data.


\subsubsection{task\_pid}
\index{task\_pid}
General syntax:

\begin{vindent}
\begin{verbatim}
task_pid:long (task:long)
\end{verbatim}
\end{vindent}
Returns the process ID of the given task.


\subsubsection{task\_prio}
\index{task\_prio}
General syntax:

\begin{vindent}
\begin{verbatim}
task_prio:long (task:long)
\end{verbatim}
\end{vindent}
Returns the priority value of the given task.


\subsubsection{task\_state}
\index{task\_state}
General syntax:

\begin{vindent}
\begin{verbatim}
task_state:long (task:long)
\end{verbatim}
\end{vindent}
Returns the state of the given task. Possible states are:

\begin{vindent}
\begin{verbatim}
TASK_RUNNING           0
TASK_INTERRUPTIBLE     1
TASK_UNINTERRUPTIBLE   2
TASK_STOPPED           4
TASK_TRACED            8
EXIT_ZOMBIE           16
EXIT_DEAD             32
\end{verbatim}
\end{vindent}

\subsubsection{task\_tid}
\index{task\_tid}
General syntax:

\begin{vindent}
\begin{verbatim}
task_tid:long (task:long)
\end{verbatim}
\end{vindent}
Returns the thread ID of the given task.


\subsubsection{task\_uid}
\index{task\_uid}
General syntax:

\begin{vindent}
\begin{verbatim}
task_uid:long (task:long)
\end{verbatim}
\end{vindent}
Returns the user ID of the given task.


\subsubsection{task\_open\_file\_handles}
\index{task\_open\_file\_handles}
General syntax:

\begin{vindent}
\begin{verbatim}
task_open_file_handles:long(task:long)
\end{verbatim}
\end{vindent}
Returns the number of open file handles for the given task.


\subsubsection{task\_max\_file\_handles}
\index{task\_max\_file\_handles}
General syntax:

\begin{vindent}
\begin{verbatim}
task_max_file_handles:long(task:long)
\end{verbatim}
\end{vindent}
Returns the maximum number of file handles for the given task.


\subsection{Accessing string data at a probe point}

The following functions provide methods to access string data at a probe
point.


\subsubsection{kernel\_string}
\index{kernel\_string}
General syntax:

\begin{vindent}
\begin{verbatim}
kernel_string:string (addr:long)
\end{verbatim}
\end{vindent}
Copies a string from kernel space at a given address. The validation of this
address is only partial.


\subsubsection{user\_string\label{sub:user_string}}
\index{user\_string}
General syntax:

\begin{vindent}
\begin{verbatim}
user_string:string (addr:long)
\end{verbatim}
\end{vindent}
This function copies a string from user space at a given address. The validation
of this address is only partial. In rare cases when userspace data is not
accessible, this function returns the string \texttt{<unknown>.}


\subsubsection{user\_string2}
\index{user\_string2}
General syntax:

\begin{vindent}
\begin{verbatim}
user_string2:string (addr:long, err_msg:string)
\end{verbatim}
\end{vindent}
This function is similar to \texttt{user\_string}, (Section~\ref{sub:user_string})
but allows passing an error message as an argument to be returned if userspace
data is not available.


\subsubsection{user\_string\_warn}
\index{user\_string\_warn}
General syntax:

\begin{vindent}
\begin{verbatim}
user_string_warn:string (addr:long)
\end{verbatim}
\end{vindent}
This function copies a string from userspace at given address. It prints
a verbose error message on failure.


\subsubsection{user\_string\_quoted}
\index{user\_string\_quoted}
General syntax:

\begin{vindent}
\begin{verbatim}
user_string_quoted:string (addr:long)
\end{verbatim}
\end{vindent}
This function copies a string from userspace at given address. Any ASCII
characters that are not printable are replaced by the corresponding escape
sequence in the returned string. 


\subsection{Initializing queue statistics}
\index{queue statistics}
The queue\_stats tapset provides functions that, when given notification
of queuing events like wait, run, or done, track averages such as queue length,
service and wait times, and utilization. Call the following three functions
from appropriate probes, in sequence.


\subsubsection{qs\_wait}
\index{qs\_wait}
General syntax:

\begin{vindent}
\begin{verbatim}
qs_wait:unknown (qname:string)
\end{verbatim}
\end{vindent}
This function records that a new request was enqueued for the given queue
name.


\subsubsection{qs\_run}
\index{qs\_run}
General syntax:

\begin{vindent}
\begin{verbatim}
qs_run:unknown (qname:string)
\end{verbatim}
\end{vindent}
This function records that a previously enqueued request was removed from
the given wait queue and is now being serviced.


\subsubsection{qs\_done}
\index{qs\_done}
General syntax:

\begin{vindent}
\begin{verbatim}
qs_done:unknown (qname:string)
\end{verbatim}
\end{vindent}
This function records that a request originally from the given queue has
completed being serviced.


\subsection{Using queue statistics}

Functions with the qsq\_ prefix query the statistics averaged since the first
queue operation or when qsq\_start was called. Since statistics are often
fractional, a scale parameter multiplies the result to a more useful scale.
For some fractions, a scale of 100 returns percentage numbers.


\subsubsection{qsq\_blocked}
\index{qsq\_blocked}
General syntax:

\begin{vindent}
\begin{verbatim}
qsq_blocked:long (qname:string, scale:long)
\end{verbatim}
\end{vindent}
This function returns the fraction of elapsed time during which one or more
requests were on the wait queue.


\subsubsection{qsq\_print}
\index{qsq\_print}
General syntax:

\begin{vindent}
\begin{verbatim}
qsq_print:unknown (qname:string)
\end{verbatim}
\end{vindent}
This function prints a line containing the following statistics for the given
queue: 

\begin{itemize}
\item queue name
\item average rate of requests per second
\item average wait queue length
\item average time on the wait queue
\item average time to service a request
\item percentage of time the wait queue was used
\item percentage of time any request was being serviced
\end{itemize}

\subsubsection{qsq\_service\_time}
\index{qsq\_service\_time}
General syntax:

\begin{vindent}
\begin{verbatim}
qsq_service_time:long (qname:string, scale:long)
\end{verbatim}
\end{vindent}
This function returns the average time in microseconds required to service
a request once it is removed from the wait queue.


\subsubsection{qsq\_start}
\index{qsq\_start}
General syntax:

\begin{vindent}
\begin{verbatim}
qsq_start:unknown (qname:string)
\end{verbatim}
\end{vindent}
This function resets the statistics counters for the given queue, and restarts
tracking from the moment the function was called. This command is used to
create a queue.


\subsubsection{qsq\_throughput}
\index{qsq\_throughput}
General syntax:

\begin{vindent}
\begin{verbatim}
qsq_throughput:long (qname:string, scale:long)
\end{verbatim}
\end{vindent}
This function returns the average number of requests served per microsecond.


\subsubsection{qsq\_utilization}
\index{qsq\_utilization}
General syntax:

\begin{vindent}
\begin{verbatim}
qsq_utilization:long (qname:string, scale:long)
\end{verbatim}
\end{vindent}
This function returns the average time in microseconds that at least one
request was being serviced.


\subsubsection{qsq\_wait\_queue\_length}
\index{qsq wait\_queue\_length}
General syntax:

\begin{vindent}
\begin{verbatim}
qsq_wait_queue_length:long (qname:string, scale:long)
\end{verbatim}
\end{vindent}
This function returns the average length of the wait queue.


\subsubsection{qsq\_wait\_time}
\index{qsq\_wait\_time}
General syntax:

\begin{vindent}
\begin{verbatim}
qsq_wait_time:long (qname:string, scale:long)
\end{verbatim}
\end{vindent}
This function returns the average time in microseconds that it took for a
request to be serviced (qs\_wait() to qs\_done()).


\subsubsection{A queue example}

What follows is an example from src/testsuite/systemtap.samples/queue\_demo.stp.
It uses the randomize feature of the timer probe to simulate queuing activity.

\begin{vindent}
\begin{verbatim}
probe begin {
    qsq_start ("block-read")
    qsq_start ("block-write")
}

probe timer.ms(3500), end {
    qsq_print ("block-read")
    qsq_start ("block-read")
    qsq_print ("block-write")
    qsq_start ("block-write")
}

probe timer.ms(10000) {
    exit ()
}

# synthesize queue work/service using three randomized "threads" for each queue.
global tc

function qs_doit (thread, name) {
    n = tc[thread] = (tc[thread]+1) % 3 # per-thread state counter
    if (n==1) qs_wait (name)
    else if (n==2) qs_run (name)
    else if (n==0) qs_done (name)
}

probe timer.ms(100).randomize(100) { qs_doit (0, "block-read") }
probe timer.ms(100).randomize(100) { qs_doit (1, "block-read") }
probe timer.ms(100).randomize(100) { qs_doit (2, "block-read") }
probe timer.ms(100).randomize(100) { qs_doit (3, "block-write") }
probe timer.ms(100).randomize(100) { qs_doit (4, "block-write") }
probe timer.ms(100).randomize(100) { qs_doit (5, "block-write") }
\end{verbatim}
\end{vindent}
This prints:

\begin{vindent}
\begin{verbatim}
block-read: 9 ops/s, 1.090 qlen, 215749 await, 96382 svctm, 69% wait, 64% util
block-write: 9 ops/s, 0.992 qlen, 208485 await, 103150 svctm, 69% wait, 61% util
block-read: 9 ops/s, 0.968 qlen, 197411 await, 97762 svctm, 63% wait, 63% util
block-write: 8 ops/s, 0.930 qlen, 202414 await, 93870 svctm, 60% wait, 56% util
block-read: 8 ops/s, 0.774 qlen, 192957 await, 99995 svctm, 58% wait, 62% util
block-write: 9 ops/s, 0.861 qlen, 193857 await, 101573 svctm, 56% wait, 64% util
\end{verbatim}
\end{vindent}

\subsection{Probe point identification}

The following functions help you identify probe points.


\subsubsection{pp}
\index{pp}
General syntax:

\begin{vindent}
\begin{verbatim}
pp:string ()
\end{verbatim}
\end{vindent}
This function returns the probe point associated with a currently running
probe handler, including alias and wild-card expansion effects.


\subsubsection{probefunc}
\index{probefunc}
General syntax:

\begin{vindent}
\begin{verbatim}
probefunc:string ()
\end{verbatim}
\end{vindent}
This function returns the name of the function being probed.


\subsubsection{probemod}
\index{probefunc}
General syntax:

\begin{vindent}
\begin{verbatim}
probemod:string ()
\end{verbatim}
\end{vindent}
This function returns the name of the module containing the probe point.


\subsection{Formatting functions}
\index{formatting}
The following functions help you format output.


\subsubsection{ctime}
\index{ctime}
General syntax:

\begin{vindent}
\begin{verbatim}
ctime:string(epochsecs:long)
\end{verbatim}
\end{vindent}
This function accepts an argument of seconds since the epoch as returned
by \texttt{gettimeofday\_s()}. It returns a date string in UTC of the form:

\begin{vindent}
\begin{verbatim}
"Wed Jun 30 21:49:008 2006"
\end{verbatim}
\end{vindent}
This function does not adjust for timezones. The returned time is always
in GMT. Your script must manually adjust epochsecs before passing it to ctime()
if you want to print local time.


\subsubsection{errno\_str}
\index{errno\_str}
General syntax:

\begin{vindent}
\begin{verbatim}
errno_str:string (err:long)
\end{verbatim}
\end{vindent}
This function returns the symbolic string associated with the given error
code, such as ENOENT for the number 2, or E\#3333 for an out-of-range value
such as 3333.


\subsubsection{returnstr}
\index{returnstr}
General syntax:

\begin{vindent}
\begin{verbatim}
returnstr:string (returnp:long)
\end{verbatim}
\end{vindent}
This function is used by the syscall tapset, and returns a string. Set \texttt{}returnp
equal to 1 for decimal, or 2 for hex.


\subsubsection{thread\_indent}
\index{thread\_indent}
General syntax:

\begin{vindent}
\begin{verbatim}
thread_indent:string (delta:long)
\end{verbatim}
\end{vindent}
This function returns a string with appropriate indentation for a thread.
Call it with a small positive or matching negative delta. If this is the
outermost, initial level of indentation, then the function resets the relative
timestamp base to zero.

The following example uses thread\_indent() to trace the functions called
in the drivers/usb/core kernel source. It prints a relative timestamp and
the name and ID of the current process, followed by the appropriate indent
and the function name. Note that \char`\"{}swapper(0)\char`\"{} indicates
the kernel is running in interrupt context and there is no valid current
process.

\begin{vindent}
\begin{verbatim}
probe kernel.function("*@drivers/usb/core/*") {
    printf ("%s -> %s\n", thread_indent(1), probefunc())
}
probe kernel.function("*@drivers/usb/core/*").return {
    printf ("%s <- %s\n", thread_indent(-1), probefunc())
}
\end{verbatim}
\end{vindent}
This prints:

\begin{vindent}
\begin{verbatim}
 0 swapper(0): -> usb_hcd_irq
 8 swapper(0): <- usb_hcd_irq
 0 swapper(0): -> usb_hcd_irq
10 swapper(0):  -> usb_hcd_giveback_urb
16 swapper(0):   -> urb_unlink
22 swapper(0):   <- urb_unlink
29 swapper(0):   -> usb_free_urb
35 swapper(0):   <- usb_free_urb
39 swapper(0):  <- usb_hcd_giveback_urb
45 swapper(0): <- usb_hcd_irq
 0 usb-storage(1338): -> usb_submit_urb
 6 usb-storage(1338):  -> usb_hcd_submit_urb
12 usb-storage(1338):   -> usb_get_urb
18 usb-storage(1338):   <- usb_get_urb
25 usb-storage(1338):  <- usb_hcd_submit_urb
29 usb-storage(1338): <- usb_submit_urb
 0 swapper(0): -> usb_hcd_irq
 7 swapper(0): <- usb_hcd_irq
\end{verbatim}
\end{vindent}

\subsubsection{thread\_timestamp}
\index{thread\_timestamp}

General syntax:

\begin{vindent}
\begin{verbatim}
thread_timestamp:long ()
\end{verbatim}
\end{vindent}
This function returns an absolute timestamp value for use by the indentation
function. The default function uses \texttt{gettimeofday\_us.}


\subsection{String functions}
\index{string}
The following are string functions you can use.


\subsubsection{isinstr}
\index{isinstr}
General syntax:

\begin{vindent}
\begin{verbatim}
isinstr:long (s1:string, s2:string)
\end{verbatim}
\end{vindent}
This function returns 1 if string s1 contains string s2, otherwise zero.


\subsubsection{strlen}
\index{strlen}
General syntax:

\begin{vindent}
\begin{verbatim}
strlen:long (str:string)
\end{verbatim}
\end{vindent}
This function returns the number of characters in str.


\subsubsection{strtol}

General syntax:

\begin{vindent}
\begin{verbatim}
strtol:long (str:string, base:long)
\end{verbatim}
\end{vindent}
This function converts the string representation of a number to an integer.
The base parameter indicates the number base to assume for the string (e.g.
16 for hex, 8 for octal, 2 for binary).


\subsubsection{substr}
\index{substr}
General syntax:

\begin{vindent}
\begin{verbatim}
substr:string (str:string, start:long, stop:long)
\end{verbatim}
\end{vindent}
This function returns the substring of \texttt{str} starting from character
position \texttt{start} and ending at character position \texttt{stop}.


\subsubsection{text\_str}
\index{text\_str}
General syntax:

\begin{vindent}
\begin{verbatim}
text_str:string (input:string)
\end{verbatim}
\end{vindent}
This function accepts a string argument. Any ASCII characters in the string
that are not printable are replaced by a corresponding escape sequence in
the returned string.


\subsubsection{text\_strn}
\index{text\_strn}
General syntax:

\begin{vindent}
\begin{verbatim}
text_strn:string (input:string, len:long, quoted:long)
\end{verbatim}
\end{vindent}
This function accepts a string of length \texttt{len}. Any ASCII characters
that are not printable are replaced by a corresponding escape sequence in
the returned string. If \texttt{quoted} is not null, the function adds a
backslash character to the output.


\subsubsection{tokenize}

General syntax:

\begin{vindent}
\begin{verbatim}
tokenize:string (input:string, delim:string)
\end{verbatim}
\end{vindent}
This function returns the next non-empty token in the given input string,
where the tokens are delimited by characters in the delim string.
If the input string is non-NULL, it returns the first token. If the input string
is NULL, it returns the next token in the string passed in the previous call
to tokenize. If no delimiter is found, the entire remaining input string
is returned.  It returns NULL when no more tokens are available.


\subsection{Timestamps}
\index{timestamps}
The following functions provide methods to extract time data.


\subsubsection{get\_cycles}
\index{get\_cycles}
General syntax:

\begin{vindent}
\begin{verbatim}
get_cycles:long ()
\end{verbatim}
\end{vindent}
This function returns the processor cycle counter value if available, else
it returns zero.


\subsubsection{gettimeofday\_ms}
\index{gettimeofday\_ms}
General syntax:

\begin{vindent}
\begin{verbatim}
gettimeofday_ms:long ()
\end{verbatim}
\end{vindent}
This function returns the number of milliseconds since the UNIX epoch.


\subsubsection{gettimeofday\_ns}
\index{gettimeofday\_ns}
General syntax:

\begin{vindent}
\begin{verbatim}
gettimeofday_ns:long ()
\end{verbatim}
\end{vindent}
This function returns the number of nanoseconds since the UNIX epoch.


\subsubsection{gettimeofday\_s}
\index{gettimeofday\_ s}
General syntax:

\begin{vindent}
\begin{verbatim}
gettimeofday_s:long ()
\end{verbatim}
\end{vindent}
This function returns the number of seconds since the UNIX epoch.


\subsubsection{gettimeofday\_us}
\index{gettimeofday\_us}
General syntax:

\begin{vindent}
\begin{verbatim}
gettimeofday_us:long ()
\end{verbatim}
\end{vindent}
This function returns the number of microseconds since the UNIX epoch.


\subsection{Miscellaneous tapset functions}

The following are miscellaneous functions.


\subsubsection{addr\_to\_node}
\index{addr\_to\_node}
General syntax:

\begin{vindent}
\begin{verbatim}
addr_to_node:long (addr:long)
\end{verbatim}
\end{vindent}
This function accepts an address, and returns the node that the given address
belongs to in a NUMA system.


\subsubsection{exit}
\index{exit}
General syntax:

\begin{vindent}
\begin{verbatim}
exit:unknown ()
\end{verbatim}
\end{vindent}
This function enqueues a request to shut down the SystemTap session. It does
not unwind the current probe handler, nor block new probe handlers. The stap
daemon will respond to the request and initiate an ordered shutdown.


\subsubsection{system}
\index{system}
General syntax:

\begin{vindent}
\begin{verbatim}
system (cmd:string)
\end{verbatim}
\end{vindent}
This function runs a command on the system. The command will run in the background
when the current probe completes.


\section{For Further Reference\label{sec:For-Further-Reference}}

For more information, see:
\begin{itemize}
\item The SystemTap tutorial at \url{http://sourceware.org/systemtap/tutorial/}
\item The SystemTap wiki at \url{http://sourceware.org/systemtap/wiki}
\item The SystemTap documentation page at \url{http://sourceware.org/systemtap/documentation.html}
\item From an unpacked source tarball or GIT directory, the examples in in the
src/examples directory, the tapsets in the src/tapset directory, and the
test scripts in the src/testsuite directory.
\item The man pages for tapsets. For a list, run the command \texttt{{}``man -k
stapprobes}''.
\end {itemize}

\setcounter{secnumdepth}{0}
\newpage{}
\addcontentsline{toc}{section}{Index}
\printindex{}
\end{document}
